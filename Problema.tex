\documentclass{article}
\usepackage[utf8]{inputenc}
\usepackage{graphicx}

\title{Juego del molino.}
\author{Andrés Felipe Florián Qutián, Andrey Javier Lizarazo Hernández.}
\date{26 de septiembre de 2019}

\renewcommand{\baselinestretch}{1.5} 

\begin{document}

\maketitle

\section{Problema a representar.}
  
\includegraphics[width=4cm, height=4cm]{juego.png}

Se implementará el juego del molino, el que consiste en dos cuadrados como el que se muestra anteriormente, el juego consiste en 12 fichas 6 blancas y 6 negras, las cuales se posicionan en cada punto, con el fin de de hacer líneas de tres. Si hay una línea de tres del mismo color ya sea blanca o negra, se quitará cualquiera de las fichas del otro color, y pierde el que quede con solo dos fichas.

Este juego presenta diferentes problemas, como cual es el movimiento más optimo y las restricciones que se presentan a la hora de jugar. 

\section{Reglas.}

\begin{itemize}
  \item Seís fichas de cada color.
  \item Se forma un molino con una línea de tres, el cual permite quitar una ficha del otro color.
  \item Se pierde cuando solo quedan dos fichas.
  \item Se puede mover una casilla por turno.
\end{itemize}

\section{Solución.}

Se quieren buscar diferentes restricciones las cuales permitan evidenciar las reglas del juego, por otro lado se quiere plantear algunos de los movimiento más optimos para el buen desearrolo del juego.

\end{document}
